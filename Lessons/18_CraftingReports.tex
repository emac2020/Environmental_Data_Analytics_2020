\documentclass[]{article}
\usepackage{lmodern}
\usepackage{amssymb,amsmath}
\usepackage{ifxetex,ifluatex}
\usepackage{fixltx2e} % provides \textsubscript
\ifnum 0\ifxetex 1\fi\ifluatex 1\fi=0 % if pdftex
  \usepackage[T1]{fontenc}
  \usepackage[utf8]{inputenc}
\else % if luatex or xelatex
  \ifxetex
    \usepackage{mathspec}
  \else
    \usepackage{fontspec}
  \fi
  \defaultfontfeatures{Ligatures=TeX,Scale=MatchLowercase}
\fi
% use upquote if available, for straight quotes in verbatim environments
\IfFileExists{upquote.sty}{\usepackage{upquote}}{}
% use microtype if available
\IfFileExists{microtype.sty}{%
\usepackage[]{microtype}
\UseMicrotypeSet[protrusion]{basicmath} % disable protrusion for tt fonts
}{}
\PassOptionsToPackage{hyphens}{url} % url is loaded by hyperref
\usepackage[unicode=true]{hyperref}
\hypersetup{
            pdftitle={17: Crafting Reports},
            pdfauthor={Environmental Data Analytics \textbar{} Kateri Salk},
            pdfborder={0 0 0},
            breaklinks=true}
\urlstyle{same}  % don't use monospace font for urls
\usepackage[margin=2.54cm]{geometry}
\usepackage{color}
\usepackage{fancyvrb}
\newcommand{\VerbBar}{|}
\newcommand{\VERB}{\Verb[commandchars=\\\{\}]}
\DefineVerbatimEnvironment{Highlighting}{Verbatim}{commandchars=\\\{\}}
% Add ',fontsize=\small' for more characters per line
\usepackage{framed}
\definecolor{shadecolor}{RGB}{248,248,248}
\newenvironment{Shaded}{\begin{snugshade}}{\end{snugshade}}
\newcommand{\KeywordTok}[1]{\textcolor[rgb]{0.13,0.29,0.53}{\textbf{#1}}}
\newcommand{\DataTypeTok}[1]{\textcolor[rgb]{0.13,0.29,0.53}{#1}}
\newcommand{\DecValTok}[1]{\textcolor[rgb]{0.00,0.00,0.81}{#1}}
\newcommand{\BaseNTok}[1]{\textcolor[rgb]{0.00,0.00,0.81}{#1}}
\newcommand{\FloatTok}[1]{\textcolor[rgb]{0.00,0.00,0.81}{#1}}
\newcommand{\ConstantTok}[1]{\textcolor[rgb]{0.00,0.00,0.00}{#1}}
\newcommand{\CharTok}[1]{\textcolor[rgb]{0.31,0.60,0.02}{#1}}
\newcommand{\SpecialCharTok}[1]{\textcolor[rgb]{0.00,0.00,0.00}{#1}}
\newcommand{\StringTok}[1]{\textcolor[rgb]{0.31,0.60,0.02}{#1}}
\newcommand{\VerbatimStringTok}[1]{\textcolor[rgb]{0.31,0.60,0.02}{#1}}
\newcommand{\SpecialStringTok}[1]{\textcolor[rgb]{0.31,0.60,0.02}{#1}}
\newcommand{\ImportTok}[1]{#1}
\newcommand{\CommentTok}[1]{\textcolor[rgb]{0.56,0.35,0.01}{\textit{#1}}}
\newcommand{\DocumentationTok}[1]{\textcolor[rgb]{0.56,0.35,0.01}{\textbf{\textit{#1}}}}
\newcommand{\AnnotationTok}[1]{\textcolor[rgb]{0.56,0.35,0.01}{\textbf{\textit{#1}}}}
\newcommand{\CommentVarTok}[1]{\textcolor[rgb]{0.56,0.35,0.01}{\textbf{\textit{#1}}}}
\newcommand{\OtherTok}[1]{\textcolor[rgb]{0.56,0.35,0.01}{#1}}
\newcommand{\FunctionTok}[1]{\textcolor[rgb]{0.00,0.00,0.00}{#1}}
\newcommand{\VariableTok}[1]{\textcolor[rgb]{0.00,0.00,0.00}{#1}}
\newcommand{\ControlFlowTok}[1]{\textcolor[rgb]{0.13,0.29,0.53}{\textbf{#1}}}
\newcommand{\OperatorTok}[1]{\textcolor[rgb]{0.81,0.36,0.00}{\textbf{#1}}}
\newcommand{\BuiltInTok}[1]{#1}
\newcommand{\ExtensionTok}[1]{#1}
\newcommand{\PreprocessorTok}[1]{\textcolor[rgb]{0.56,0.35,0.01}{\textit{#1}}}
\newcommand{\AttributeTok}[1]{\textcolor[rgb]{0.77,0.63,0.00}{#1}}
\newcommand{\RegionMarkerTok}[1]{#1}
\newcommand{\InformationTok}[1]{\textcolor[rgb]{0.56,0.35,0.01}{\textbf{\textit{#1}}}}
\newcommand{\WarningTok}[1]{\textcolor[rgb]{0.56,0.35,0.01}{\textbf{\textit{#1}}}}
\newcommand{\AlertTok}[1]{\textcolor[rgb]{0.94,0.16,0.16}{#1}}
\newcommand{\ErrorTok}[1]{\textcolor[rgb]{0.64,0.00,0.00}{\textbf{#1}}}
\newcommand{\NormalTok}[1]{#1}
\usepackage{longtable,booktabs}
% Fix footnotes in tables (requires footnote package)
\IfFileExists{footnote.sty}{\usepackage{footnote}\makesavenoteenv{long table}}{}
\usepackage{graphicx,grffile}
\makeatletter
\def\maxwidth{\ifdim\Gin@nat@width>\linewidth\linewidth\else\Gin@nat@width\fi}
\def\maxheight{\ifdim\Gin@nat@height>\textheight\textheight\else\Gin@nat@height\fi}
\makeatother
% Scale images if necessary, so that they will not overflow the page
% margins by default, and it is still possible to overwrite the defaults
% using explicit options in \includegraphics[width, height, ...]{}
\setkeys{Gin}{width=\maxwidth,height=\maxheight,keepaspectratio}
\IfFileExists{parskip.sty}{%
\usepackage{parskip}
}{% else
\setlength{\parindent}{0pt}
\setlength{\parskip}{6pt plus 2pt minus 1pt}
}
\setlength{\emergencystretch}{3em}  % prevent overfull lines
\providecommand{\tightlist}{%
  \setlength{\itemsep}{0pt}\setlength{\parskip}{0pt}}
\setcounter{secnumdepth}{0}
% Redefines (sub)paragraphs to behave more like sections
\ifx\paragraph\undefined\else
\let\oldparagraph\paragraph
\renewcommand{\paragraph}[1]{\oldparagraph{#1}\mbox{}}
\fi
\ifx\subparagraph\undefined\else
\let\oldsubparagraph\subparagraph
\renewcommand{\subparagraph}[1]{\oldsubparagraph{#1}\mbox{}}
\fi

% set default figure placement to htbp
\makeatletter
\def\fps@figure{htbp}
\makeatother


\title{17: Crafting Reports}
\author{Environmental Data Analytics \textbar{} Kateri Salk}
\date{Spring 2019}

\begin{document}
\maketitle

\subsection{LESSON OBJECTIVES}\label{lesson-objectives}

\begin{enumerate}
\def\labelenumi{\arabic{enumi}.}
\tightlist
\item
  Describe the purpose of using R Markdown as a communication and
  workflow tool
\item
  Incorporate Markdown syntax into documents
\item
  Communicate the process and findings of an analysis session in the
  style of a report
\end{enumerate}

\subsection{BASIC R MARKDOWN DOCUMENT
STRUCTURE}\label{basic-r-markdown-document-structure}

\begin{enumerate}
\def\labelenumi{\arabic{enumi}.}
\tightlist
\item
  \textbf{YAML Header} surrounded by --- on top and bottom

  \begin{itemize}
  \tightlist
  \item
    YAML templates include options for html, pdf, word, markdown, and
    interactive
  \item
    More information on formatting the YAML header can be found in the
    cheat sheet Header info:
  \item
    optional: author, date, geometry
  \item
    want to include specific output (pdf, html, word, interactive
    options, more in cheatsheets below)
  \end{itemize}
\item
  \textbf{R Code Chunks} surrounded by
  ``\texttt{on\ top\ and\ bottom\ +\ Create\ using}Cmd/Ctrl\texttt{+}Alt\texttt{+}I`

  \begin{itemize}
  \tightlist
  \item
    Can be named \{r name\} to facilitate navigation and autoreferencing
  \item
    Chunk options allow for flexibility when the code runs and when the
    document is knitted Notes:
  \end{itemize}
\item
  \textbf{Text} with formatting options for readability in knitted
  document
\end{enumerate}

A handy cheat sheet for R markdown can be found
\href{https://www.rstudio.com/wp-content/uploads/2015/03/rmarkdown-reference.pdf}{here}.
Another one can be found
\href{https://www.rstudio.com/wp-content/uploads/2015/02/rmarkdown-cheatsheet.pdf}{here}.

\subsection{WHY R MARKDOWN?}\label{why-r-markdown}

 ( anything in \textless{}\ldots{}\textgreater{} don't show up in
knitted doc)

\begin{itemize}
\tightlist
\item
  \emph{italic} \# can see different color for italic vs.~bold
\item
  \textbf{bold}
\item
  \textbf{Documentation} of data pipeline steps
\item
  Store \emph{code}, \emph{visulations}, and \emph{text} in one place
\item
  \textbf{Organize} code into sections
\item
  Aesthetically pleasing
\item
  Manage changes in a \emph{reproducible} fashion
\end{itemize}

\subsection{TEXT EDITING CHALLENGE}\label{text-editing-challenge}

Create a table below that details the example datasets we have been
using in class. The first column should contain the names of the
datasets and the second column should include some relevant information
about the datasets. (Hint: use the cheat sheets to figure out how to
make a table in Rmd)

Dataset Name \textbar{} Dataset Notes
\_\_\_\_\_\_\_\_\_\_\_\_\_\_\_\_\textbar{}\_\_\_\_\_\_\_\_\_\_\_\_\_\_\_\_
\textbar{} Peter Paul Lake \textbar{} Pollution Cell 3 \textbar{} Cell 4

\subsection{R CHUNK EDITING CHALLENGE}\label{r-chunk-editing-challenge}

\subsubsection{Installing packages}\label{installing-packages}

Create an R chunk below that installs the package \texttt{knitr}.
Instead of commenting out the code, customize the chunk options such
that the code is not evaluated (i.e., not run).

\begin{itemize}
\item ~
  \subsection{creates headings that can be
  searched}\label{creates-headings-that-can-be-searched}
\end{itemize}

\subsubsection{Setup}\label{setup}

Create an R chunk below called ``setup'' that checks your working
directory, loads the packages \texttt{tidyverse} and \texttt{knitr}, and
sets a ggplot theme. Remember that you need to disable R throwing a
message, which containts a check mark that cannot be knitted.

Load the NTL-LTER\_Lake\_Nutrients\_Raw dataset, display the head of the
dataset, and set the date column to a date format.

Customize the chunk options such that the code is run but is not
displayed in the final document.

\begin{verbatim}
## [1] "/Users/emilymcnamara/Desktop/Env Data Analytics/Environmental_Data_Analytics_2020"
\end{verbatim}

\begin{verbatim}
##   lakeid  lakename year4 daynum sampledate depth_id depth tn_ug tp_ug nh34 no23
## 1      L Paul Lake  1991    140    5/20/91        1  0.00   538    25   NA   NA
## 2      L Paul Lake  1991    140    5/20/91        2  0.85   285    14   NA   NA
## 3      L Paul Lake  1991    140    5/20/91        3  1.75   399    14   NA   NA
## 4      L Paul Lake  1991    140    5/20/91        4  3.00   453    14   NA   NA
## 5      L Paul Lake  1991    140    5/20/91        5  4.00   363    13   NA   NA
## 6      L Paul Lake  1991    140    5/20/91        6  6.00   583    37   NA   NA
##   po4 comments
## 1  NA         
## 2  NA         
## 3  NA         
## 4  NA         
## 5  NA         
## 6  NA
\end{verbatim}

\subsubsection{Data Exploration, Wrangling, and
Visualization}\label{data-exploration-wrangling-and-visualization}

Create an R chunk below to create a processed dataset do the following
operations:

\begin{itemize}
\tightlist
\item
  Include all columns except lakeid, depth\_id, and comments
\item
  Include only surface samples (depth = 0 m)
\end{itemize}

Create a second R chunk to create a summary dataset with the mean,
minimum, maximum, and standard deviation of total nitrogen
concentrations for each lake. Create a second summary dataset that is
identical except that it evaluates total phosphorus. Customize the chunk
options such that the code is run but not displayed in the final
document.

Create a third R chunk that uses the function \texttt{kable} in the
knitr package to display two tables: one for the summary dataframe for
total N and one for the summary dataframe of total P. Use the
\texttt{caption\ =\ "\ "} code within that function to title your
tables. Customize the chunk options such that the final table is
displayed but not the code used to generate the table.

\begin{longtable}[]{@{}lrrrr@{}}
\caption{Total N Summary Data}\tabularnewline
\toprule
lakename & meanTN & minTN & maxTN & sdTN\tabularnewline
\midrule
\endfirsthead
\toprule
lakename & meanTN & minTN & maxTN & sdTN\tabularnewline
\midrule
\endhead
Central Long Lake & 675.8338 & 343.020 & 953.063 &
203.25838\tabularnewline
Crampton Lake & 362.6813 & 353.380 & 376.304 & 12.05748\tabularnewline
East Long Lake & 796.0141 & 299.310 & 3316.892 &
413.16241\tabularnewline
Hummingbird Lake & 1036.6695 & 779.053 & 1221.960 &
204.36889\tabularnewline
Paul Lake & 365.0679 & 45.670 & 628.625 & 107.31194\tabularnewline
Peter Lake & 548.2733 & 131.830 & 2048.151 & 320.83105\tabularnewline
Tuesday Lake & 410.2571 & 237.363 & 554.418 & 71.53021\tabularnewline
West Long Lake & 737.5649 & 303.170 & 2950.343 &
437.18914\tabularnewline
\bottomrule
\end{longtable}

\begin{longtable}[]{@{}lrrrr@{}}
\caption{Total P Summary Data}\tabularnewline
\toprule
lakename & meanTP & minTP & maxTP & sdTP\tabularnewline
\midrule
\endfirsthead
\toprule
lakename & meanTP & minTP & maxTP & sdTP\tabularnewline
\midrule
\endhead
Central Long Lake & 21.59478 & 8.190 & 37.270 & 7.067785\tabularnewline
Crampton Lake & 11.16033 & 5.803 & 15.555 & 4.946759\tabularnewline
East Long Lake & 39.16260 & 7.160 & 136.671 & 29.814346\tabularnewline
Hummingbird Lake & 36.21925 & 32.765 & 42.119 & 4.146717\tabularnewline
Paul Lake & 10.27233 & 0.110 & 36.070 & 4.507096\tabularnewline
Peter Lake & 21.33466 & 0.000 & 66.893 & 14.110973\tabularnewline
Tuesday Lake & 11.37472 & 4.413 & 18.663 & 3.076367\tabularnewline
West Long Lake & 22.34388 & 2.690 & 63.243 & 12.622100\tabularnewline
\bottomrule
\end{longtable}

Create a fourth and fifth R chunk that generates two plots (one in each
chunk): one for total N over time with different colors for each lake,
and one with the same setup but for total P. Decide which geom option
will be appropriate for your purpose, and select a color palette that is
visually pleasing and accessible. Customize the chunk options such that
the final figures are displayed but not the code used to generate the
figures. In addition, customize the chunk options such that the figures
are aligned on the left side of the page. Lastly, add a fig.cap chunk
option to add a caption (title) to your plot that will display
underneath the figure.

\begin{Shaded}
\begin{Highlighting}[]
\CommentTok{# Total N Plot}

\NormalTok{TNplot <-}\StringTok{ }
\StringTok{  }\KeywordTok{ggplot}\NormalTok{(NTL.nutrient.data, }\KeywordTok{aes}\NormalTok{(}\DataTypeTok{x =}\NormalTok{ sampledate, }\DataTypeTok{y =}\NormalTok{ tn_ug, }
                     \DataTypeTok{color =}\NormalTok{ lakename)) }\OperatorTok{+}
\StringTok{  }\KeywordTok{geom_point}\NormalTok{() }\OperatorTok{+}
\StringTok{  }\KeywordTok{labs}\NormalTok{(}\DataTypeTok{x =} \KeywordTok{expression}\NormalTok{(}\KeywordTok{paste}\NormalTok{(}\StringTok{"Date"}\NormalTok{)),}
    \DataTypeTok{y =} \KeywordTok{expression}\NormalTok{(}\KeywordTok{paste}\NormalTok{(}\StringTok{"TN ( "}\NormalTok{, mu,}\StringTok{"g/L)"}\NormalTok{))) }\OperatorTok{+}
\StringTok{  }\KeywordTok{scale_color_brewer}\NormalTok{(}\DataTypeTok{palette =} \StringTok{"Spectral"}\NormalTok{) }\OperatorTok{+}
\StringTok{  }\NormalTok{mytheme}
\KeywordTok{print}\NormalTok{(TNplot)}
\end{Highlighting}
\end{Shaded}

\begin{verbatim}
## Warning: Removed 139 rows containing missing values (geom_point).
\end{verbatim}

\begin{figure}
\centering
\includegraphics{18_CraftingReports_files/figure-latex/Figure 1-1.pdf}
\caption{Total N over time}
\end{figure}

\begin{Shaded}
\begin{Highlighting}[]
\CommentTok{# Total P Plot}

\NormalTok{TPplot <-}\StringTok{ }
\StringTok{  }\KeywordTok{ggplot}\NormalTok{(NTL.nutrient.data, }\KeywordTok{aes}\NormalTok{(}\DataTypeTok{x =}\NormalTok{ sampledate, }\DataTypeTok{y =}\NormalTok{ tp_ug, }
                     \DataTypeTok{color =}\NormalTok{ lakename)) }\OperatorTok{+}
\StringTok{  }\KeywordTok{geom_point}\NormalTok{() }\OperatorTok{+}
\StringTok{  }\KeywordTok{labs}\NormalTok{(}\DataTypeTok{x =} \KeywordTok{expression}\NormalTok{(}\KeywordTok{paste}\NormalTok{(}\StringTok{"Date"}\NormalTok{)),}
    \DataTypeTok{y =} \KeywordTok{expression}\NormalTok{(}\KeywordTok{paste}\NormalTok{(}\StringTok{"TP ( "}\NormalTok{, mu,}\StringTok{"g/L)"}\NormalTok{))) }\OperatorTok{+}
\StringTok{  }\KeywordTok{scale_color_brewer}\NormalTok{(}\DataTypeTok{palette =} \StringTok{"Spectral"}\NormalTok{) }\OperatorTok{+}
\StringTok{  }\NormalTok{mytheme}
\KeywordTok{print}\NormalTok{(TPplot)}
\end{Highlighting}
\end{Shaded}

\begin{verbatim}
## Warning: Removed 7 rows containing missing values (geom_point).
\end{verbatim}

\begin{figure}
\centering
\includegraphics{18_CraftingReports_files/figure-latex/Figure 2-1.pdf}
\caption{Total P over time}
\end{figure}

\subsubsection{Other options}\label{other-options}

What are the chunk options that will suppress the display of errors,
warnings, and messages in the final document?

\begin{quote}
ANSWER:
\end{quote}

\subsubsection{Communicating results}\label{communicating-results}

Write a paragraph describing your findings from the R coding challenge
above. This should be geared toward an educated audience but one that is
not necessarily familiar with the dataset. Then insert a horizontal rule
below the paragraph. Below the horizontal rule, write another paragraph
describing the next steps you might take in analyzing this dataset. What
questions might you be able to answer, and what analyses would you
conduct to answer those questions?

\subsection{KNIT YOUR PDF}\label{knit-your-pdf}

When you have completed the above steps, try knitting your PDF to see if
all of the formatting options you specified turned out as planned. This
may take some troubleshooting.

\subsection{OTHER R MARKDOWN CUSTOMIZATION
OPTIONS}\label{other-r-markdown-customization-options}

We have covered the basics in class today, but R Markdown offers many
customization options. A word of caution: customizing templates will
often require more interaction with LaTeX and installations on your
computer, so be ready to troubleshoot issues.

Customization options for pdf output include:

\begin{itemize}
\tightlist
\item
  Table of contents
\item
  Number sections
\item
  Control default size of figures
\item
  Citations
\item
  Template (more info
  \href{http://jianghao.wang/post/2017-12-08-rmarkdown-templates/}{here})
\end{itemize}

pdf\_document:\\
toc: true\\
number\_sections: true\\
fig\_height: 3\\
fig\_width: 4\\
citation\_package: natbib\\
template:

\end{document}
