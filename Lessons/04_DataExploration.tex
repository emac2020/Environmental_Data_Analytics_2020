\documentclass[]{article}
\usepackage{lmodern}
\usepackage{amssymb,amsmath}
\usepackage{ifxetex,ifluatex}
\usepackage{fixltx2e} % provides \textsubscript
\ifnum 0\ifxetex 1\fi\ifluatex 1\fi=0 % if pdftex
  \usepackage[T1]{fontenc}
  \usepackage[utf8]{inputenc}
\else % if luatex or xelatex
  \ifxetex
    \usepackage{mathspec}
  \else
    \usepackage{fontspec}
  \fi
  \defaultfontfeatures{Ligatures=TeX,Scale=MatchLowercase}
\fi
% use upquote if available, for straight quotes in verbatim environments
\IfFileExists{upquote.sty}{\usepackage{upquote}}{}
% use microtype if available
\IfFileExists{microtype.sty}{%
\usepackage[]{microtype}
\UseMicrotypeSet[protrusion]{basicmath} % disable protrusion for tt fonts
}{}
\PassOptionsToPackage{hyphens}{url} % url is loaded by hyperref
\usepackage[unicode=true]{hyperref}
\hypersetup{
            pdftitle={4: Data Exploration},
            pdfauthor={Environmental Data Analytics \textbar{} Kateri Salk},
            pdfborder={0 0 0},
            breaklinks=true}
\urlstyle{same}  % don't use monospace font for urls
\usepackage[margin=2.54cm]{geometry}
\usepackage{color}
\usepackage{fancyvrb}
\newcommand{\VerbBar}{|}
\newcommand{\VERB}{\Verb[commandchars=\\\{\}]}
\DefineVerbatimEnvironment{Highlighting}{Verbatim}{commandchars=\\\{\}}
% Add ',fontsize=\small' for more characters per line
\usepackage{framed}
\definecolor{shadecolor}{RGB}{248,248,248}
\newenvironment{Shaded}{\begin{snugshade}}{\end{snugshade}}
\newcommand{\KeywordTok}[1]{\textcolor[rgb]{0.13,0.29,0.53}{\textbf{#1}}}
\newcommand{\DataTypeTok}[1]{\textcolor[rgb]{0.13,0.29,0.53}{#1}}
\newcommand{\DecValTok}[1]{\textcolor[rgb]{0.00,0.00,0.81}{#1}}
\newcommand{\BaseNTok}[1]{\textcolor[rgb]{0.00,0.00,0.81}{#1}}
\newcommand{\FloatTok}[1]{\textcolor[rgb]{0.00,0.00,0.81}{#1}}
\newcommand{\ConstantTok}[1]{\textcolor[rgb]{0.00,0.00,0.00}{#1}}
\newcommand{\CharTok}[1]{\textcolor[rgb]{0.31,0.60,0.02}{#1}}
\newcommand{\SpecialCharTok}[1]{\textcolor[rgb]{0.00,0.00,0.00}{#1}}
\newcommand{\StringTok}[1]{\textcolor[rgb]{0.31,0.60,0.02}{#1}}
\newcommand{\VerbatimStringTok}[1]{\textcolor[rgb]{0.31,0.60,0.02}{#1}}
\newcommand{\SpecialStringTok}[1]{\textcolor[rgb]{0.31,0.60,0.02}{#1}}
\newcommand{\ImportTok}[1]{#1}
\newcommand{\CommentTok}[1]{\textcolor[rgb]{0.56,0.35,0.01}{\textit{#1}}}
\newcommand{\DocumentationTok}[1]{\textcolor[rgb]{0.56,0.35,0.01}{\textbf{\textit{#1}}}}
\newcommand{\AnnotationTok}[1]{\textcolor[rgb]{0.56,0.35,0.01}{\textbf{\textit{#1}}}}
\newcommand{\CommentVarTok}[1]{\textcolor[rgb]{0.56,0.35,0.01}{\textbf{\textit{#1}}}}
\newcommand{\OtherTok}[1]{\textcolor[rgb]{0.56,0.35,0.01}{#1}}
\newcommand{\FunctionTok}[1]{\textcolor[rgb]{0.00,0.00,0.00}{#1}}
\newcommand{\VariableTok}[1]{\textcolor[rgb]{0.00,0.00,0.00}{#1}}
\newcommand{\ControlFlowTok}[1]{\textcolor[rgb]{0.13,0.29,0.53}{\textbf{#1}}}
\newcommand{\OperatorTok}[1]{\textcolor[rgb]{0.81,0.36,0.00}{\textbf{#1}}}
\newcommand{\BuiltInTok}[1]{#1}
\newcommand{\ExtensionTok}[1]{#1}
\newcommand{\PreprocessorTok}[1]{\textcolor[rgb]{0.56,0.35,0.01}{\textit{#1}}}
\newcommand{\AttributeTok}[1]{\textcolor[rgb]{0.77,0.63,0.00}{#1}}
\newcommand{\RegionMarkerTok}[1]{#1}
\newcommand{\InformationTok}[1]{\textcolor[rgb]{0.56,0.35,0.01}{\textbf{\textit{#1}}}}
\newcommand{\WarningTok}[1]{\textcolor[rgb]{0.56,0.35,0.01}{\textbf{\textit{#1}}}}
\newcommand{\AlertTok}[1]{\textcolor[rgb]{0.94,0.16,0.16}{#1}}
\newcommand{\ErrorTok}[1]{\textcolor[rgb]{0.64,0.00,0.00}{\textbf{#1}}}
\newcommand{\NormalTok}[1]{#1}
\usepackage{graphicx,grffile}
\makeatletter
\def\maxwidth{\ifdim\Gin@nat@width>\linewidth\linewidth\else\Gin@nat@width\fi}
\def\maxheight{\ifdim\Gin@nat@height>\textheight\textheight\else\Gin@nat@height\fi}
\makeatother
% Scale images if necessary, so that they will not overflow the page
% margins by default, and it is still possible to overwrite the defaults
% using explicit options in \includegraphics[width, height, ...]{}
\setkeys{Gin}{width=\maxwidth,height=\maxheight,keepaspectratio}
\IfFileExists{parskip.sty}{%
\usepackage{parskip}
}{% else
\setlength{\parindent}{0pt}
\setlength{\parskip}{6pt plus 2pt minus 1pt}
}
\setlength{\emergencystretch}{3em}  % prevent overfull lines
\providecommand{\tightlist}{%
  \setlength{\itemsep}{0pt}\setlength{\parskip}{0pt}}
\setcounter{secnumdepth}{0}
% Redefines (sub)paragraphs to behave more like sections
\ifx\paragraph\undefined\else
\let\oldparagraph\paragraph
\renewcommand{\paragraph}[1]{\oldparagraph{#1}\mbox{}}
\fi
\ifx\subparagraph\undefined\else
\let\oldsubparagraph\subparagraph
\renewcommand{\subparagraph}[1]{\oldsubparagraph{#1}\mbox{}}
\fi

% set default figure placement to htbp
\makeatletter
\def\fps@figure{htbp}
\makeatother


\title{4: Data Exploration}
\author{Environmental Data Analytics \textbar{} Kateri Salk}
\date{Spring 2020}

\begin{document}
\maketitle

\subsection{Lesson Objectives}\label{lesson-objectives}

\begin{enumerate}
\def\labelenumi{\arabic{enumi}.}
\tightlist
\item
  Set up a data analysis session in RStudio
\item
  Import and explore datasets in R
\item
  Apply data exploration skills to a real-world example dataset
\end{enumerate}

\subsection{Best Practices in R}\label{best-practices-in-r}

In many situations in data analytics, you may be expected to work from
multiple computers or share projects among multiple users. A few general
best practices will avoid common pitfalls related to collaborative work.

\subsubsection{Set your working
directory}\label{set-your-working-directory}

A session in RStudio will always function by mapping to a specific
folder in your computer, called the \emph{working directory}. All
navigation between folders and files will happen relative to this
working directory. When you open an R project, your working directory
will automatically set to the folder that holds the project file. If you
open an R script or RMarkdown document directly by double-clicking the
file, your working directory will automatically set to the folder that
holds that file. It is a good idea to note with a comment at the top of
your file which working directory you intend the user to designate.

In this course, we will always open the R project file for the course,
and additional navigation of the working directory will happen from that
folder. To check your working directory, use the following R command:

\begin{Shaded}
\begin{Highlighting}[]
\CommentTok{# Working directory should be set to the parent folder for the Environmental Data Analytics Course, i.e., the folder that houses the R Project file.}

\KeywordTok{getwd}\NormalTok{()}
\end{Highlighting}
\end{Shaded}

\begin{verbatim}
## [1] "/Users/emilymcnamara/Desktop/Env Data Analytics/Environmental_Data_Analytics_2020"
\end{verbatim}

\begin{Shaded}
\begin{Highlighting}[]
\CommentTok{# Asks R to tell you what it has set as your working directory }
\end{Highlighting}
\end{Shaded}

If your working directory is not set to the folder you want, you have
several options. The first is to directly code your working directory.
You may do this by defining an absolute file path (below). What are the
pitfalls of using an absolute file path?

\begin{Shaded}
\begin{Highlighting}[]
\CommentTok{# Absolute file path is commented out}
\CommentTok{#setwd("/Users/katerisalk/Documents/Duke/Courses/Environmental_Data_Analytics")}

\CommentTok{# This allows you to set a specific working directory as one of your choosing }
\CommentTok{# Can also go to "session" and "set working directory" and can choose a location}
\CommentTok{# Can comment to yourself that this script will be set to a certain file path and if it's not set to that, this is where it should be...}
\end{Highlighting}
\end{Shaded}

You may change your working directory without coding by going to the
Session menu in RStudio and navigating to the Set Working Directory tab.
From there, you may select from a series of options to reset your
working directory.

Another option is to use the R package \texttt{here}. We will not be
using this option in class, but it is growing quite popular among R
users. A more detailed description and rationale can be found here:
\url{https://github.com/jennybc/here_here}.

\subsubsection{Load your packages}\label{load-your-packages}

At the top of your R scripts, you should load any packages that need to
be used for that R script. A common issue that arises is that packages
will be loaded in the middle of the code, making it difficult to run
specific chunks of code without scrolling to make sure all necessary
packages are loaded. For example, the tidyverse package is one that we
will use regularly in class.

At the same time, you should also load your theme if you are doing any
data visualization with ggplot. More on this later.

\begin{Shaded}
\begin{Highlighting}[]
\CommentTok{# Load package}
\KeywordTok{library}\NormalTok{(tidyverse)}

\CommentTok{# Allows you to see which packages you've loaded and where there may be some conflicts}
\CommentTok{# Have to write "message = FALSE" to ensure that document can be knitted so the warning messages aren't printed out}
\end{Highlighting}
\end{Shaded}

\subsubsection{Import your datasets}\label{import-your-datasets}

Datasets can be imported into R. Good data practices dictate that raw
data (from yourself or others) should not be changed and re-saved within
the spreadsheet, but rather the data should be changed with reproducible
techniques and saved as a new file. Note: data should be saved in
nonproprietary formats, namely .csv or .txt files rather than .xls or
.xlsx files.

We're going to be using csv files.

To read in a data file, you may specify a file path with an
\emph{absolute} or a \emph{relative} file path. As above with your
working directory, it is a better practice to use a relative directory.
To navigate a relative file path, use \texttt{./} followed by the tab
key to navigate forward in the folder structure, and use \texttt{../}
followed by the tab key to navigate back out of the folder structure.
For example, this lesson is located in the ``Lessons'' folder, and we
need to navigate into the ``Data'' folder. After clicking the correct
folder, use \texttt{/} and press tab again to continue the process.

You may also import datasets from the Files tab, but this is not
recommended since this is not reproducible.

\begin{Shaded}
\begin{Highlighting}[]
\CommentTok{# Absolute file path (not recommended)}
\CommentTok{#read.csv("/Users/katerisalk/Documents/Duke/Courses/Environmental_Data_Analytics/Data/Raw/USGS_Site02085000_Flow_Raw.csv")}

\CommentTok{# MetaData file in Data folder includes specific info for the corresponding raw data files}
\CommentTok{# If have multiple tabs in an excel file, separate tabs and save each one as csv file to upload it}

\CommentTok{# Relative file path (friendly for users regardless of machine)}
\NormalTok{USGS.flow.data <-}\StringTok{ }\KeywordTok{read.csv}\NormalTok{(}\StringTok{"./Data/Raw/USGS_Site02085000_Flow_Raw.csv"}\NormalTok{)}

\CommentTok{# Because we know R is pointing to the project file that contains our data, we can have it search for a specific data file in our project}
\CommentTok{# starting with "." means "go one folder in"}
\CommentTok{# Do ".." to mean go out when folder and then in another folder}
\CommentTok{# Even though working directory is sent to Project, Knit directory may not be, so have to set Knit directory to project directory (knit --> knit directory --> project directory)}
\CommentTok{# Can also go directly to folder on desktop and copy the path name of the specific data file}

\CommentTok{# What happens if we don't assign a name to our imported dataset?}
\CommentTok{#read.csv("./Data/Raw/USGS_Site02085000_Flow_Raw.csv")}
\NormalTok{## If it's not named, R just reads the data and tells you what it said. It won't be called up in Environment and all the data will appear in a knitted pdf}

\CommentTok{# Another option is to choose with your browser}
\CommentTok{#read.csv(file.choose())}
\CommentTok{# pulls up finder to choose specific folder BUT not reproducible because others can't tell which folder you're pulling from }

\CommentTok{# To import .txt files, use read.table rather than read.csv}
\CommentTok{#read.table()}
\end{Highlighting}
\end{Shaded}

\subsection{EXPLORE YOUR DATASET}\label{explore-your-dataset}

Take a moment to read through the README file associated with the USGS
dataset on discharge at the Eno River. Where can you find this file? How
does the placement and information found in this file relate to the best
practices for reproducible data analysis? \textgreater{} ANSWER: You
find it under Data --\textgreater{} Metadata. It provides descriptions
and clarity to what abbreviations and numeric codes mean. It also
provides the source, so people can go directly to the data to navigate
it through the main page. Shows date of access to, so if it was accessed
on an earlier date, people can see if the data was revised more
recently. Includes info on how to save the data in a way that is
consistent and others can find it: Files are named according to the
following naming convention:
\texttt{databasename\_datatype\_details\_stage.format}, where: (see meta
data info at bottom of doc)

\begin{Shaded}
\begin{Highlighting}[]
\KeywordTok{View}\NormalTok{(USGS.flow.data)}
\end{Highlighting}
\end{Shaded}

\begin{verbatim}
## Warning in system2("/usr/bin/otool", c("-L", shQuote(DSO)), stdout = TRUE):
## running command ''/usr/bin/otool' -L '/Library/Frameworks/R.framework/Resources/
## modules/R_de.so'' had status 1
\end{verbatim}

\begin{Shaded}
\begin{Highlighting}[]
\CommentTok{# Alternate option: click on data frame in Environment tab}

\KeywordTok{class}\NormalTok{(USGS.flow.data)}
\end{Highlighting}
\end{Shaded}

\begin{verbatim}
## [1] "data.frame"
\end{verbatim}

\begin{Shaded}
\begin{Highlighting}[]
\CommentTok{# Tells you the class which is "data frame". This means its 2D and that it contains different modes "numeric and factors"}
\KeywordTok{colnames}\NormalTok{(USGS.flow.data)}
\end{Highlighting}
\end{Shaded}

\begin{verbatim}
##  [1] "agency_cd"              "site_no"                "datetime"              
##  [4] "X165986_00060_00001"    "X165986_00060_00001_cd" "X165987_00060_00002"   
##  [7] "X165987_00060_00002_cd" "X84936_00060_00003"     "X84936_00060_00003_cd" 
## [10] "X84937_00065_00001"     "X84937_00065_00001_cd"  "X84938_00065_00002"    
## [13] "X84938_00065_00002_cd"  "X84939_00065_00003"     "X84939_00065_00003_cd"
\end{verbatim}

\begin{Shaded}
\begin{Highlighting}[]
\CommentTok{# Not very informative}

\CommentTok{# Rename columns}
\KeywordTok{colnames}\NormalTok{(USGS.flow.data) <-}\StringTok{ }\KeywordTok{c}\NormalTok{(}\StringTok{"agency_cd"}\NormalTok{, }\StringTok{"site_no"}\NormalTok{, }\StringTok{"datetime"}\NormalTok{, }
                              \StringTok{"discharge.max"}\NormalTok{, }\StringTok{"discharge.max.approval"}\NormalTok{, }
                              \StringTok{"discharge.min"}\NormalTok{, }\StringTok{"discharge.min.approval"}\NormalTok{, }
                              \StringTok{"discharge.mean"}\NormalTok{, }\StringTok{"discharge.mean.approval"}\NormalTok{, }
                              \StringTok{"gage.height.max"}\NormalTok{, }\StringTok{"gage.height.max.approval"}\NormalTok{, }
                              \StringTok{"gage.height.min"}\NormalTok{, }\StringTok{"gage.height.min.approval"}\NormalTok{, }
                              \StringTok{"gage.height.mean"}\NormalTok{, }\StringTok{"gage.height.mean.approval"}\NormalTok{)}
\CommentTok{# Have to list all the column names, can't just list specific ones. Got these column names from Meta Data. By writing "c" it says: I want to concatinate a list of these names. Inserting returns gives you clarity so you can see if there's something wrong.}
\CommentTok{# This only saves the renamed columns of dataframe in R, not in excel. Need to specifically code it to change it in excel.}
\CommentTok{# General coding rule: Don't surpass 80 characters in a single line. Can see character count on bottom left of R script.}
\CommentTok{# If want to specify, write: colnames(USGS.flow.data)[] and put specific names in there. You can put "3" in the brackets to name a specific column }

\KeywordTok{str}\NormalTok{(USGS.flow.data)}
\end{Highlighting}
\end{Shaded}

\begin{verbatim}
## 'data.frame':    33690 obs. of  15 variables:
##  $ agency_cd                : Factor w/ 1 level "USGS": 1 1 1 1 1 1 1 1 1 1 ...
##  $ site_no                  : int  2085000 2085000 2085000 2085000 2085000 2085000 2085000 2085000 2085000 2085000 ...
##  $ datetime                 : Factor w/ 33690 levels "1/1/00","1/1/01",..: 2873 3896 4919 5198 5291 5384 5477 5570 5663 2966 ...
##  $ discharge.max            : num  NA NA NA NA NA NA NA NA NA NA ...
##  $ discharge.max.approval   : Factor w/ 3 levels "","A","P": 1 1 1 1 1 1 1 1 1 1 ...
##  $ discharge.min            : num  NA NA NA NA NA NA NA NA NA NA ...
##  $ discharge.min.approval   : Factor w/ 3 levels "","A","P": 1 1 1 1 1 1 1 1 1 1 ...
##  $ discharge.mean           : num  39 39 39 39 39 39 39 39 39 39 ...
##  $ discharge.mean.approval  : Factor w/ 4 levels "","A","A:e","P": 2 2 2 2 2 2 2 2 2 2 ...
##  $ gage.height.max          : num  NA NA NA NA NA NA NA NA NA NA ...
##  $ gage.height.max.approval : Factor w/ 3 levels "","A","P": 1 1 1 1 1 1 1 1 1 1 ...
##  $ gage.height.min          : num  NA NA NA NA NA NA NA NA NA NA ...
##  $ gage.height.min.approval : Factor w/ 3 levels "","A","P": 1 1 1 1 1 1 1 1 1 1 ...
##  $ gage.height.mean         : num  NA NA NA NA NA NA NA NA NA NA ...
##  $ gage.height.mean.approval: Factor w/ 3 levels "","A","P": 1 1 1 1 1 1 1 1 1 1 ...
\end{verbatim}

\begin{Shaded}
\begin{Highlighting}[]
\CommentTok{# Structure of a data frame. Tells us: you have a data frame and then the size of the data frame and gives a summary of each of the columns. }
\CommentTok{# shows some columns as N/A and some as just blanks. I.e. Discharge max is numeric so it's N/A, Discharge max approval is factors/characters that are blanked and will be retained as blanks.}

\KeywordTok{dim}\NormalTok{(USGS.flow.data)}
\end{Highlighting}
\end{Shaded}

\begin{verbatim}
## [1] 33690    15
\end{verbatim}

\begin{Shaded}
\begin{Highlighting}[]
\CommentTok{# Dimensions of the data frame. Can also see this info in the Environment tab and from structure.}

\KeywordTok{length}\NormalTok{(USGS.flow.data)}
\end{Highlighting}
\end{Shaded}

\begin{verbatim}
## [1] 15
\end{verbatim}

\begin{Shaded}
\begin{Highlighting}[]
\CommentTok{# Length gives the number of columns NOT the number of rows}

\KeywordTok{nrow}\NormalTok{(USGS.flow.data)}
\end{Highlighting}
\end{Shaded}

\begin{verbatim}
## [1] 33690
\end{verbatim}

\begin{Shaded}
\begin{Highlighting}[]
\CommentTok{# Gives the number of rows}

\KeywordTok{head}\NormalTok{(USGS.flow.data)}
\end{Highlighting}
\end{Shaded}

\begin{verbatim}
##   agency_cd site_no datetime discharge.max discharge.max.approval discharge.min
## 1      USGS 2085000  10/1/27            NA                                   NA
## 2      USGS 2085000  10/2/27            NA                                   NA
## 3      USGS 2085000  10/3/27            NA                                   NA
## 4      USGS 2085000  10/4/27            NA                                   NA
## 5      USGS 2085000  10/5/27            NA                                   NA
## 6      USGS 2085000  10/6/27            NA                                   NA
##   discharge.min.approval discharge.mean discharge.mean.approval gage.height.max
## 1                                    39                       A              NA
## 2                                    39                       A              NA
## 3                                    39                       A              NA
## 4                                    39                       A              NA
## 5                                    39                       A              NA
## 6                                    39                       A              NA
##   gage.height.max.approval gage.height.min gage.height.min.approval
## 1                                       NA                         
## 2                                       NA                         
## 3                                       NA                         
## 4                                       NA                         
## 5                                       NA                         
## 6                                       NA                         
##   gage.height.mean gage.height.mean.approval
## 1               NA                          
## 2               NA                          
## 3               NA                          
## 4               NA                          
## 5               NA                          
## 6               NA
\end{verbatim}

\begin{Shaded}
\begin{Highlighting}[]
\CommentTok{# Gives idea of what the first 6 rows look like}
\KeywordTok{head}\NormalTok{(USGS.flow.data, }\DecValTok{10}\NormalTok{)}
\end{Highlighting}
\end{Shaded}

\begin{verbatim}
##    agency_cd site_no datetime discharge.max discharge.max.approval
## 1       USGS 2085000  10/1/27            NA                       
## 2       USGS 2085000  10/2/27            NA                       
## 3       USGS 2085000  10/3/27            NA                       
## 4       USGS 2085000  10/4/27            NA                       
## 5       USGS 2085000  10/5/27            NA                       
## 6       USGS 2085000  10/6/27            NA                       
## 7       USGS 2085000  10/7/27            NA                       
## 8       USGS 2085000  10/8/27            NA                       
## 9       USGS 2085000  10/9/27            NA                       
## 10      USGS 2085000 10/10/27            NA                       
##    discharge.min discharge.min.approval discharge.mean discharge.mean.approval
## 1             NA                                    39                       A
## 2             NA                                    39                       A
## 3             NA                                    39                       A
## 4             NA                                    39                       A
## 5             NA                                    39                       A
## 6             NA                                    39                       A
## 7             NA                                    39                       A
## 8             NA                                    39                       A
## 9             NA                                    39                       A
## 10            NA                                    39                       A
##    gage.height.max gage.height.max.approval gage.height.min
## 1               NA                                       NA
## 2               NA                                       NA
## 3               NA                                       NA
## 4               NA                                       NA
## 5               NA                                       NA
## 6               NA                                       NA
## 7               NA                                       NA
## 8               NA                                       NA
## 9               NA                                       NA
## 10              NA                                       NA
##    gage.height.min.approval gage.height.mean gage.height.mean.approval
## 1                                         NA                          
## 2                                         NA                          
## 3                                         NA                          
## 4                                         NA                          
## 5                                         NA                          
## 6                                         NA                          
## 7                                         NA                          
## 8                                         NA                          
## 9                                         NA                          
## 10                                        NA
\end{verbatim}

\begin{Shaded}
\begin{Highlighting}[]
\CommentTok{# Gives first 10 rows}
\KeywordTok{tail}\NormalTok{(USGS.flow.data, }\DecValTok{5}\NormalTok{)}
\end{Highlighting}
\end{Shaded}

\begin{verbatim}
##       agency_cd site_no datetime discharge.max discharge.max.approval
## 33686      USGS 2085000 12/22/19            NA                       
## 33687      USGS 2085000 12/23/19            NA                       
## 33688      USGS 2085000 12/24/19            NA                       
## 33689      USGS 2085000 12/25/19            NA                       
## 33690      USGS 2085000 12/26/19            NA                       
##       discharge.min discharge.min.approval discharge.mean
## 33686            NA                                  18.1
## 33687            NA                                  18.6
## 33688            NA                                  18.8
## 33689            NA                                  16.6
## 33690            NA                                  15.1
##       discharge.mean.approval gage.height.max gage.height.max.approval
## 33686                       P              NA                         
## 33687                       P              NA                         
## 33688                       P              NA                         
## 33689                       P              NA                         
## 33690                       P              NA                         
##       gage.height.min gage.height.min.approval gage.height.mean
## 33686              NA                                      1.93
## 33687              NA                                      1.94
## 33688              NA                                      1.95
## 33689              NA                                      1.91
## 33690              NA                                      1.88
##       gage.height.mean.approval
## 33686                         P
## 33687                         P
## 33688                         P
## 33689                         P
## 33690                         P
\end{verbatim}

\begin{Shaded}
\begin{Highlighting}[]
\CommentTok{# Gives the last 5 rows}
\NormalTok{USGS.flow.data[}\DecValTok{30000}\OperatorTok{:}\DecValTok{30005}\NormalTok{, }\KeywordTok{c}\NormalTok{(}\DecValTok{3}\NormalTok{, }\DecValTok{8}\NormalTok{, }\DecValTok{14}\NormalTok{)]}
\end{Highlighting}
\end{Shaded}

\begin{verbatim}
##       datetime discharge.mean gage.height.mean
## 30000 11/18/09           27.5             1.72
## 30001 11/19/09           31.6             1.80
## 30002 11/20/09           37.1             1.88
## 30003 11/21/09           32.1             1.80
## 30004 11/22/09           23.7             1.66
## 30005 11/23/09          337.0             3.87
\end{verbatim}

\begin{Shaded}
\begin{Highlighting}[]
\CommentTok{# Matrix subsetting: lines 30,000 to 30,000 and specific column and row by a specific amount}

\KeywordTok{class}\NormalTok{(USGS.flow.data}\OperatorTok{$}\NormalTok{datetime)}
\end{Highlighting}
\end{Shaded}

\begin{verbatim}
## [1] "factor"
\end{verbatim}

\begin{Shaded}
\begin{Highlighting}[]
\CommentTok{# Can ask for specific columns. I.e. specific date and time by just writing USGS.flow.data$datetime without "class"}
\CommentTok{# Adding "class" tells you which class it is}
\KeywordTok{class}\NormalTok{(USGS.flow.data}\OperatorTok{$}\NormalTok{discharge.mean)}
\end{Highlighting}
\end{Shaded}

\begin{verbatim}
## [1] "numeric"
\end{verbatim}

\begin{Shaded}
\begin{Highlighting}[]
\KeywordTok{class}\NormalTok{(USGS.flow.data}\OperatorTok{$}\NormalTok{gage.height.mean)}
\end{Highlighting}
\end{Shaded}

\begin{verbatim}
## [1] "numeric"
\end{verbatim}

\begin{Shaded}
\begin{Highlighting}[]
\KeywordTok{summary}\NormalTok{(USGS.flow.data)}
\end{Highlighting}
\end{Shaded}

\begin{verbatim}
##  agency_cd       site_no           datetime     discharge.max    
##  USGS:33690   Min.   :2085000   1/1/00 :    1   Min.   :   0.26  
##               1st Qu.:2085000   1/1/01 :    1   1st Qu.:   7.23  
##               Median :2085000   1/1/02 :    1   Median :  21.15  
##               Mean   :2085000   1/1/03 :    1   Mean   :  88.15  
##               3rd Qu.:2085000   1/1/04 :    1   3rd Qu.:  59.80  
##               Max.   :2085000   1/1/05 :    1   Max.   :4730.00  
##                                 (Other):33684   NA's   :28342    
##  discharge.max.approval discharge.min     discharge.min.approval
##   :28342                Min.   :   0.09    :28342               
##  A: 5347                1st Qu.:   4.38   A: 5347               
##  P:    1                Median :  12.60   P:    1               
##                         Mean   :  30.46                         
##                         3rd Qu.:  34.80                         
##                         Max.   :1460.00                         
##                         NA's   :28342                           
##  discharge.mean    discharge.mean.approval gage.height.max 
##  Min.   :   0.02      : 5108               Min.   : 0.890  
##  1st Qu.:   9.30   A  :28265               1st Qu.: 1.490  
##  Median :  24.00   A:e:  294               Median : 1.830  
##  Mean   :  59.48   P  :   23               Mean   : 2.124  
##  3rd Qu.:  54.00                           3rd Qu.: 2.310  
##  Max.   :4600.00                           Max.   :17.020  
##  NA's   :5108                              NA's   :28229   
##  gage.height.max.approval gage.height.min gage.height.min.approval
##   :28229                  Min.   :0.840    :28229                 
##  A: 5460                  1st Qu.:1.380   A: 5460                 
##  P:    1                  Median :1.650   P:    1                 
##                           Mean   :1.736                           
##                           3rd Qu.:2.030                           
##                           Max.   :9.190                           
##                           NA's   :28229                           
##  gage.height.mean gage.height.mean.approval
##  Min.   : 0.870    :24870                  
##  1st Qu.: 1.450   A: 8797                  
##  Median : 1.770   P:   23                  
##  Mean   : 1.952                            
##  3rd Qu.: 2.200                            
##  Max.   :15.040                            
##  NA's   :24870
\end{verbatim}

\begin{Shaded}
\begin{Highlighting}[]
\CommentTok{# Structure can be more interprative. Factors won't appear as conveniently as numeric values}
\end{Highlighting}
\end{Shaded}

What happened to blank cells in the spreadsheet when they were imported
into R? \textgreater{} Answer:

\subsection{Adjusting Datasets}\label{adjusting-datasets}

\subsubsection{Removing NAs}\label{removing-nas}

Notice in our dataset that our discharge and gage height observations
have many NAs, meaning no measurement was recorded for a specific day.
In some cases, it might be in our best interest to remove NAs from a
dataset. Removing NAs or not will depend on your research question.

\begin{Shaded}
\begin{Highlighting}[]
\KeywordTok{summary}\NormalTok{(USGS.flow.data}\OperatorTok{$}\NormalTok{discharge.mean)}
\end{Highlighting}
\end{Shaded}

\begin{verbatim}
##    Min. 1st Qu.  Median    Mean 3rd Qu.    Max.    NA's 
##    0.02    9.30   24.00   59.48   54.00 4600.00    5108
\end{verbatim}

\begin{Shaded}
\begin{Highlighting}[]
\KeywordTok{summary}\NormalTok{(USGS.flow.data}\OperatorTok{$}\NormalTok{gage.height.mean)}
\end{Highlighting}
\end{Shaded}

\begin{verbatim}
##    Min. 1st Qu.  Median    Mean 3rd Qu.    Max.    NA's 
##   0.870   1.450   1.770   1.952   2.200  15.040   24870
\end{verbatim}

\begin{Shaded}
\begin{Highlighting}[]
\CommentTok{# Gives distribution and tells you how many N/As are in the data frame. Quite a few in this case}
\end{Highlighting}
\end{Shaded}

Question: What types of research questions might make it favorable to
remove NAs from a dataset, and what types of research questions might
make it favorable to retain NAs in the dataset?

\begin{quote}
Answer: Helpful to retain for future reference. Removing it can help
when running analysis on presence data instead of presence/absence. If
want to compare discharge mean and height mean, only want to do it where
values are present. Also, if want to run a correlation matrix, NAs screw
it up so need to remove it.
\end{quote}

\begin{Shaded}
\begin{Highlighting}[]
\NormalTok{USGS.flow.data.complete <-}\StringTok{ }\KeywordTok{na.omit}\NormalTok{(USGS.flow.data)}
\CommentTok{# If making permenant changes to data frame, want to call it something else or else you have to rerun code to get back to original}
\CommentTok{# na.omit only runs the complete datasets in your data frame}
\KeywordTok{dim}\NormalTok{(USGS.flow.data)}
\end{Highlighting}
\end{Shaded}

\begin{verbatim}
## [1] 33690    15
\end{verbatim}

\begin{Shaded}
\begin{Highlighting}[]
\KeywordTok{dim}\NormalTok{(USGS.flow.data.complete)}
\end{Highlighting}
\end{Shaded}

\begin{verbatim}
## [1] 5342   15
\end{verbatim}

\begin{Shaded}
\begin{Highlighting}[]
\CommentTok{# A lot fewer rows without NAs}

\KeywordTok{mean}\NormalTok{(USGS.flow.data.complete}\OperatorTok{$}\NormalTok{discharge.mean)}
\end{Highlighting}
\end{Shaded}

\begin{verbatim}
## [1] 51.08613
\end{verbatim}

\begin{Shaded}
\begin{Highlighting}[]
\KeywordTok{sd}\NormalTok{(USGS.flow.data.complete}\OperatorTok{$}\NormalTok{discharge.mean)}
\end{Highlighting}
\end{Shaded}

\begin{verbatim}
## [1] 137.2094
\end{verbatim}

\begin{Shaded}
\begin{Highlighting}[]
\KeywordTok{summary}\NormalTok{(USGS.flow.data.complete}\OperatorTok{$}\NormalTok{discharge.mean)}
\end{Highlighting}
\end{Shaded}

\begin{verbatim}
##     Min.  1st Qu.   Median     Mean  3rd Qu.     Max. 
##    0.220    5.683   16.600   51.086   44.800 3270.000
\end{verbatim}

\begin{Shaded}
\begin{Highlighting}[]
\CommentTok{# These are means, sd's, and summary of the complete dataset without the NAs}
\CommentTok{# Summary doesn't show sd, so need to run that separately }
\end{Highlighting}
\end{Shaded}

\subsubsection{Formatting dates}\label{formatting-dates}

R will often import dates as factors or characters rather than dates. To
fix, this we need to tell R that it is looking at dates. We also need to
specify the format the dates are in. By default, if you don't provide a
format, R will attempt to use \%Y-\%m-\%d or \%Y/\%m/\%d as a default.
Note: if you are working collaboratively in an international setting,
using a year-month-day format in spreadsheets is the least ambiguous of
date formats. Make sure to check whether month-day-year or
day-month-year is used in an ambiguously formatted spreadsheet.

Formatting of dates in R:

\%d day as number (0-31) \%m month (00-12, can be e.g., 01 or 1) \%y
2-digit year \%Y 4-digit year \%a abbreviated weekday \%A unabbreviated
weekday \%b abbreviated month \%B unabbreviated month

In some cases when dates are provided as integers, you may need to
provide an origin for your dates. Beware: the ``origin'' date for Excel
(Windows), Excel (Mac), R, and MATLAB all have different origin dates.
Google this if it comes up. Origin will be January 1st, 1970

\begin{Shaded}
\begin{Highlighting}[]
\KeywordTok{help}\NormalTok{(as.Date)}

\CommentTok{# Adjust date formatting for today}
\CommentTok{# Write code for three different date formats. }
\CommentTok{# An example is provided to get you started.}
\CommentTok{# (code must be uncommented)}
\NormalTok{today <-}\StringTok{ }\KeywordTok{Sys.Date}\NormalTok{()}
\CommentTok{# Whatever day your computer thinks it is will be the date that's called up. It shows up in the environment and is always: year-month-day}

\KeywordTok{format}\NormalTok{(today, }\DataTypeTok{format =} \StringTok{"%B"}\NormalTok{)}
\end{Highlighting}
\end{Shaded}

\begin{verbatim}
## [1] "January"
\end{verbatim}

\begin{Shaded}
\begin{Highlighting}[]
\CommentTok{#Formats today as the month}

\KeywordTok{format}\NormalTok{(today, }\DataTypeTok{format =} \StringTok{"%d"}\NormalTok{)}
\end{Highlighting}
\end{Shaded}

\begin{verbatim}
## [1] "21"
\end{verbatim}

\begin{Shaded}
\begin{Highlighting}[]
\KeywordTok{format}\NormalTok{(today, }\DataTypeTok{format =} \StringTok{"%b"}\NormalTok{)}
\end{Highlighting}
\end{Shaded}

\begin{verbatim}
## [1] "Jan"
\end{verbatim}

\begin{Shaded}
\begin{Highlighting}[]
\KeywordTok{format}\NormalTok{(today, }\DataTypeTok{format =} \StringTok{"%y"}\NormalTok{)}
\end{Highlighting}
\end{Shaded}

\begin{verbatim}
## [1] "20"
\end{verbatim}

\begin{Shaded}
\begin{Highlighting}[]
\NormalTok{USGS.flow.data}\OperatorTok{$}\NormalTok{datetime <-}\StringTok{ }\KeywordTok{as.Date}\NormalTok{(USGS.flow.data}\OperatorTok{$}\NormalTok{datetime, }\DataTypeTok{format =} \StringTok{"%m/%d/%y"}\NormalTok{)}
\CommentTok{# If want R to perceive date/time column as date, have to tell R which column, and which format exists in the data right now rather than the format we want it in}
\CommentTok{#BUT it says it's 2027 instead of 1997 because all the dates prior to 1969 are noted in the future}
\end{Highlighting}
\end{Shaded}

Note that for every date prior to 1969, R has assigned the date in the
2000s rather than the 1900s. This can be fixed with an \texttt{ifelse}
statement inside a function. Run through the code below and write what
is happening in the comment above each line.

\begin{Shaded}
\begin{Highlighting}[]
\CommentTok{# }
\NormalTok{USGS.flow.data}\OperatorTok{$}\NormalTok{datetime <-}\StringTok{ }\KeywordTok{format}\NormalTok{(USGS.flow.data}\OperatorTok{$}\NormalTok{datetime, }\StringTok{"%y%m%d"}\NormalTok{)}

\CommentTok{#}
\NormalTok{create.early.dates <-}\StringTok{ }\NormalTok{(}\ControlFlowTok{function}\NormalTok{(d) \{}
       \KeywordTok{paste0}\NormalTok{(}\KeywordTok{ifelse}\NormalTok{(d }\OperatorTok{>}\StringTok{ }\DecValTok{181231}\NormalTok{,}\StringTok{"19"}\NormalTok{,}\StringTok{"20"}\NormalTok{),d)}
\NormalTok{       \})}
\CommentTok{#}
\NormalTok{USGS.flow.data}\OperatorTok{$}\NormalTok{datetime <-}\StringTok{ }\KeywordTok{create.early.dates}\NormalTok{(USGS.flow.data}\OperatorTok{$}\NormalTok{datetime)}

\CommentTok{#}
\NormalTok{USGS.flow.data}\OperatorTok{$}\NormalTok{datetime <-}\StringTok{ }\KeywordTok{as.Date}\NormalTok{(USGS.flow.data}\OperatorTok{$}\NormalTok{datetime, }\DataTypeTok{format =} \StringTok{"%Y%m%d"}\NormalTok{) }
\end{Highlighting}
\end{Shaded}

\subsection{Saving datasets}\label{saving-datasets}

We just edited our raw dataset into a processed form. We may want to
return to this processed dataset later, which will be easier to do if we
save it as a spreadsheet.

\begin{Shaded}
\begin{Highlighting}[]
\KeywordTok{write.csv}\NormalTok{(USGS.flow.data, }\DataTypeTok{file =} \StringTok{"./Data/Processed/USGS_Site02085000_Flow_Processed.csv"}\NormalTok{, }\DataTypeTok{row.names=}\OtherTok{FALSE}\NormalTok{)}

\CommentTok{# Want to store apart from raw data file. Have to make a folder in R Data folder named "Processed" in order to save it there.}
\end{Highlighting}
\end{Shaded}

\subsection{Tips and Tricks}\label{tips-and-tricks}

\subsubsection{Knitting}\label{knitting}

\begin{itemize}
\tightlist
\item
  In the Knit menu in the Editor, you will need to specify whether your
  knit directory should be the document directory or the project
  directory. If your document is not knitting correctly, try switching
  between the document directory and project directory as a first
  troubleshooting option.
\end{itemize}

\subsubsection{Spreadsheets}\label{spreadsheets}

*Files should be saved as .csv or .txt for easy import into R. Note that
complex formatting, including formulas in Excel, are not saved when
spreadsheets are converted to comma separated or text formats (i.e.,
values alone are saved).

*The first row is reserved for column headers.

*A secondary row for column headers (e.g., units) should not be used if
data are being imported into R. Incorporate units into the first row
column headers if necessary.

*Short names are preferred for column headers, to the extent they are
informative. Additional information can be stored in comments within R
scripts and/or in README files.

*Spaces in column names will be replaced with a \texttt{.} when imported
into R. When designing spreadsheets, avoid spaces in column headers.

*Avoid symbols in column headers. This can cause issues when importing
into R.

\end{document}
