\documentclass[]{article}
\usepackage{lmodern}
\usepackage{amssymb,amsmath}
\usepackage{ifxetex,ifluatex}
\usepackage{fixltx2e} % provides \textsubscript
\ifnum 0\ifxetex 1\fi\ifluatex 1\fi=0 % if pdftex
  \usepackage[T1]{fontenc}
  \usepackage[utf8]{inputenc}
\else % if luatex or xelatex
  \ifxetex
    \usepackage{mathspec}
  \else
    \usepackage{fontspec}
  \fi
  \defaultfontfeatures{Ligatures=TeX,Scale=MatchLowercase}
\fi
% use upquote if available, for straight quotes in verbatim environments
\IfFileExists{upquote.sty}{\usepackage{upquote}}{}
% use microtype if available
\IfFileExists{microtype.sty}{%
\usepackage[]{microtype}
\UseMicrotypeSet[protrusion]{basicmath} % disable protrusion for tt fonts
}{}
\PassOptionsToPackage{hyphens}{url} % url is loaded by hyperref
\usepackage[unicode=true]{hyperref}
\hypersetup{
            pdftitle={3: Coding Basics},
            pdfauthor={Environmental Data Analytics \textbar{} Kateri Salk},
            pdfborder={0 0 0},
            breaklinks=true}
\urlstyle{same}  % don't use monospace font for urls
\usepackage[margin=2.54cm]{geometry}
\usepackage{color}
\usepackage{fancyvrb}
\newcommand{\VerbBar}{|}
\newcommand{\VERB}{\Verb[commandchars=\\\{\}]}
\DefineVerbatimEnvironment{Highlighting}{Verbatim}{commandchars=\\\{\}}
% Add ',fontsize=\small' for more characters per line
\usepackage{framed}
\definecolor{shadecolor}{RGB}{248,248,248}
\newenvironment{Shaded}{\begin{snugshade}}{\end{snugshade}}
\newcommand{\KeywordTok}[1]{\textcolor[rgb]{0.13,0.29,0.53}{\textbf{#1}}}
\newcommand{\DataTypeTok}[1]{\textcolor[rgb]{0.13,0.29,0.53}{#1}}
\newcommand{\DecValTok}[1]{\textcolor[rgb]{0.00,0.00,0.81}{#1}}
\newcommand{\BaseNTok}[1]{\textcolor[rgb]{0.00,0.00,0.81}{#1}}
\newcommand{\FloatTok}[1]{\textcolor[rgb]{0.00,0.00,0.81}{#1}}
\newcommand{\ConstantTok}[1]{\textcolor[rgb]{0.00,0.00,0.00}{#1}}
\newcommand{\CharTok}[1]{\textcolor[rgb]{0.31,0.60,0.02}{#1}}
\newcommand{\SpecialCharTok}[1]{\textcolor[rgb]{0.00,0.00,0.00}{#1}}
\newcommand{\StringTok}[1]{\textcolor[rgb]{0.31,0.60,0.02}{#1}}
\newcommand{\VerbatimStringTok}[1]{\textcolor[rgb]{0.31,0.60,0.02}{#1}}
\newcommand{\SpecialStringTok}[1]{\textcolor[rgb]{0.31,0.60,0.02}{#1}}
\newcommand{\ImportTok}[1]{#1}
\newcommand{\CommentTok}[1]{\textcolor[rgb]{0.56,0.35,0.01}{\textit{#1}}}
\newcommand{\DocumentationTok}[1]{\textcolor[rgb]{0.56,0.35,0.01}{\textbf{\textit{#1}}}}
\newcommand{\AnnotationTok}[1]{\textcolor[rgb]{0.56,0.35,0.01}{\textbf{\textit{#1}}}}
\newcommand{\CommentVarTok}[1]{\textcolor[rgb]{0.56,0.35,0.01}{\textbf{\textit{#1}}}}
\newcommand{\OtherTok}[1]{\textcolor[rgb]{0.56,0.35,0.01}{#1}}
\newcommand{\FunctionTok}[1]{\textcolor[rgb]{0.00,0.00,0.00}{#1}}
\newcommand{\VariableTok}[1]{\textcolor[rgb]{0.00,0.00,0.00}{#1}}
\newcommand{\ControlFlowTok}[1]{\textcolor[rgb]{0.13,0.29,0.53}{\textbf{#1}}}
\newcommand{\OperatorTok}[1]{\textcolor[rgb]{0.81,0.36,0.00}{\textbf{#1}}}
\newcommand{\BuiltInTok}[1]{#1}
\newcommand{\ExtensionTok}[1]{#1}
\newcommand{\PreprocessorTok}[1]{\textcolor[rgb]{0.56,0.35,0.01}{\textit{#1}}}
\newcommand{\AttributeTok}[1]{\textcolor[rgb]{0.77,0.63,0.00}{#1}}
\newcommand{\RegionMarkerTok}[1]{#1}
\newcommand{\InformationTok}[1]{\textcolor[rgb]{0.56,0.35,0.01}{\textbf{\textit{#1}}}}
\newcommand{\WarningTok}[1]{\textcolor[rgb]{0.56,0.35,0.01}{\textbf{\textit{#1}}}}
\newcommand{\AlertTok}[1]{\textcolor[rgb]{0.94,0.16,0.16}{#1}}
\newcommand{\ErrorTok}[1]{\textcolor[rgb]{0.64,0.00,0.00}{\textbf{#1}}}
\newcommand{\NormalTok}[1]{#1}
\usepackage{graphicx,grffile}
\makeatletter
\def\maxwidth{\ifdim\Gin@nat@width>\linewidth\linewidth\else\Gin@nat@width\fi}
\def\maxheight{\ifdim\Gin@nat@height>\textheight\textheight\else\Gin@nat@height\fi}
\makeatother
% Scale images if necessary, so that they will not overflow the page
% margins by default, and it is still possible to overwrite the defaults
% using explicit options in \includegraphics[width, height, ...]{}
\setkeys{Gin}{width=\maxwidth,height=\maxheight,keepaspectratio}
\IfFileExists{parskip.sty}{%
\usepackage{parskip}
}{% else
\setlength{\parindent}{0pt}
\setlength{\parskip}{6pt plus 2pt minus 1pt}
}
\setlength{\emergencystretch}{3em}  % prevent overfull lines
\providecommand{\tightlist}{%
  \setlength{\itemsep}{0pt}\setlength{\parskip}{0pt}}
\setcounter{secnumdepth}{0}
% Redefines (sub)paragraphs to behave more like sections
\ifx\paragraph\undefined\else
\let\oldparagraph\paragraph
\renewcommand{\paragraph}[1]{\oldparagraph{#1}\mbox{}}
\fi
\ifx\subparagraph\undefined\else
\let\oldsubparagraph\subparagraph
\renewcommand{\subparagraph}[1]{\oldsubparagraph{#1}\mbox{}}
\fi

% set default figure placement to htbp
\makeatletter
\def\fps@figure{htbp}
\makeatother


\title{3: Coding Basics}
\author{Environmental Data Analytics \textbar{} Kateri Salk}
\date{Spring 2020}

\begin{document}
\maketitle

\subsection{Objectives}\label{objectives}

\begin{enumerate}
\def\labelenumi{\arabic{enumi}.}
\tightlist
\item
  Discuss and navigate different data types in R
\item
  Create, manipulate, and explore datasets
\item
  Call packages in R
\end{enumerate}

\subsection{Data Types in R}\label{data-types-in-r}

R treats objects differently based on their characteristics. For more
information, please see:
\url{https://www.statmethods.net/input/datatypes.html}.

\begin{itemize}
\item
  \textbf{Vectors} 1 dimensional structure that contains elements of the
  same type.
\item
  \textbf{Matrices} 2 dimensional structure that contains elements of
  the same type.
\item
  \textbf{Arrays} Similar to matrices, but can have more than 2
  dimensions. We will not delve into arrays in depth.
\item
  \textbf{Lists} Ordered collection of elements that can have different
  modes.
\item
  \textbf{Data Frames} 2 dimensional structure that is more general than
  a matrix. Columns can have different modes (e.g., numeric and factor).
  When we import csv files into the R workspace, they will enter as data
  frames.
\end{itemize}

Define what each new piece of syntax does below (i.e., fill in blank
comments). Note that the R chunk has been divided into sections (\# at
beginning of line, ---- at end)

\begin{Shaded}
\begin{Highlighting}[]
\CommentTok{# Vectors ----}
\NormalTok{vector1 <-}\StringTok{ }\KeywordTok{c}\NormalTok{(}\DecValTok{1}\NormalTok{,}\DecValTok{2}\NormalTok{,}\FloatTok{5.3}\NormalTok{,}\DecValTok{6}\NormalTok{,}\OperatorTok{-}\DecValTok{2}\NormalTok{,}\DecValTok{4}\NormalTok{) }\CommentTok{# numeric vector}
\NormalTok{vector1}
\end{Highlighting}
\end{Shaded}

\begin{verbatim}
## [1]  1.0  2.0  5.3  6.0 -2.0  4.0
\end{verbatim}

\begin{Shaded}
\begin{Highlighting}[]
\NormalTok{vector2 <-}\StringTok{ }\KeywordTok{c}\NormalTok{(}\StringTok{"one"}\NormalTok{,}\StringTok{"two"}\NormalTok{,}\StringTok{"three"}\NormalTok{) }\CommentTok{# character vector}
\NormalTok{vector2}
\end{Highlighting}
\end{Shaded}

\begin{verbatim}
## [1] "one"   "two"   "three"
\end{verbatim}

\begin{Shaded}
\begin{Highlighting}[]
\NormalTok{vector3 <-}\StringTok{ }\KeywordTok{c}\NormalTok{(}\OtherTok{TRUE}\NormalTok{,}\OtherTok{TRUE}\NormalTok{,}\OtherTok{TRUE}\NormalTok{,}\OtherTok{FALSE}\NormalTok{,}\OtherTok{TRUE}\NormalTok{,}\OtherTok{FALSE}\NormalTok{) }\CommentTok{#logical vector}
\NormalTok{vector3}
\end{Highlighting}
\end{Shaded}

\begin{verbatim}
## [1]  TRUE  TRUE  TRUE FALSE  TRUE FALSE
\end{verbatim}

\begin{Shaded}
\begin{Highlighting}[]
\NormalTok{vector1[}\DecValTok{3}\NormalTok{] }
\end{Highlighting}
\end{Shaded}

\begin{verbatim}
## [1] 5.3
\end{verbatim}

\begin{Shaded}
\begin{Highlighting}[]
\CommentTok{#Line 39 is saying that the third item is 5.3. [] allow you to call up the 3rd item}
\CommentTok{# "c" means concatenate, i.e. takes everything in parenthesis and names it. entering vector1 will allow you to call up the data points after you name it.}
\CommentTok{# have to put parenthesis around specific character strings or R will think you have to do an operation called one. }

\CommentTok{# Matrices ----}
\CommentTok{#going to 2 dimensions. The ones below are functions because R knows that the word matrix means make a matrix}
\NormalTok{matrix1 <-}\StringTok{ }\KeywordTok{matrix}\NormalTok{(}\DecValTok{1}\OperatorTok{:}\DecValTok{20}\NormalTok{, }\DataTypeTok{nrow =} \DecValTok{5}\NormalTok{,}\DataTypeTok{ncol =} \DecValTok{4}\NormalTok{) }\CommentTok{# has numbers 1-20 in it, with 5 rows and 4 columns}
\NormalTok{matrix1}
\end{Highlighting}
\end{Shaded}

\begin{verbatim}
##      [,1] [,2] [,3] [,4]
## [1,]    1    6   11   16
## [2,]    2    7   12   17
## [3,]    3    8   13   18
## [4,]    4    9   14   19
## [5,]    5   10   15   20
\end{verbatim}

\begin{Shaded}
\begin{Highlighting}[]
\NormalTok{matrix2 <-}\StringTok{ }\KeywordTok{matrix}\NormalTok{(}\DecValTok{1}\OperatorTok{:}\DecValTok{20}\NormalTok{, }\DataTypeTok{nrow =} \DecValTok{5}\NormalTok{, }\DataTypeTok{ncol =} \DecValTok{4}\NormalTok{, }\DataTypeTok{byrow =} \OtherTok{TRUE}\NormalTok{) }\CommentTok{#"byrow" means responses are by rows instead of columns}
\NormalTok{matrix2}
\end{Highlighting}
\end{Shaded}

\begin{verbatim}
##      [,1] [,2] [,3] [,4]
## [1,]    1    2    3    4
## [2,]    5    6    7    8
## [3,]    9   10   11   12
## [4,]   13   14   15   16
## [5,]   17   18   19   20
\end{verbatim}

\begin{Shaded}
\begin{Highlighting}[]
\NormalTok{matrix3 <-}\StringTok{ }\KeywordTok{matrix}\NormalTok{(}\DecValTok{1}\OperatorTok{:}\DecValTok{20}\NormalTok{, }\DataTypeTok{nrow =} \DecValTok{5}\NormalTok{, }\DataTypeTok{ncol =} \DecValTok{4}\NormalTok{, }\DataTypeTok{byrow =} \OtherTok{TRUE}\NormalTok{, }\CommentTok{# return after comma continues the line}
                  \DataTypeTok{dimnames =} \KeywordTok{list}\NormalTok{(}\KeywordTok{c}\NormalTok{(}\StringTok{"uno"}\NormalTok{, }\StringTok{"dos"}\NormalTok{, }\StringTok{"tres"}\NormalTok{, }\StringTok{"cuatro"}\NormalTok{, }\StringTok{"cinco"}\NormalTok{), }
                                  \KeywordTok{c}\NormalTok{(}\StringTok{"un"}\NormalTok{, }\StringTok{"deux"}\NormalTok{, }\StringTok{"trois"}\NormalTok{, }\StringTok{"cat"}\NormalTok{))) }\CommentTok{#dimnames names rows by whatever you want them to be called. this is calls the columns by french numbers 1-4 and the columns spanish numbers 1-4 }

\NormalTok{matrix1[}\DecValTok{4}\NormalTok{, ] }\CommentTok{# calls up matrix 1, row 4}
\end{Highlighting}
\end{Shaded}

\begin{verbatim}
## [1]  4  9 14 19
\end{verbatim}

\begin{Shaded}
\begin{Highlighting}[]
\NormalTok{matrix1[ , }\DecValTok{3}\NormalTok{] }\CommentTok{# calls up matrix 1, column 3}
\end{Highlighting}
\end{Shaded}

\begin{verbatim}
## [1] 11 12 13 14 15
\end{verbatim}

\begin{Shaded}
\begin{Highlighting}[]
\NormalTok{matrix1[}\KeywordTok{c}\NormalTok{(}\DecValTok{12}\NormalTok{, }\DecValTok{14}\NormalTok{)] }\CommentTok{# calls up matrix 1, numbers 12 and 14. specifying specific objects within matrix}
\end{Highlighting}
\end{Shaded}

\begin{verbatim}
## [1] 12 14
\end{verbatim}

\begin{Shaded}
\begin{Highlighting}[]
\NormalTok{matrix1[}\KeywordTok{c}\NormalTok{(}\DecValTok{12}\OperatorTok{:}\DecValTok{14}\NormalTok{)] }\CommentTok{# calls up matrix 1, numbers 12 through 14. specifying specific objects within matrix}
\end{Highlighting}
\end{Shaded}

\begin{verbatim}
## [1] 12 13 14
\end{verbatim}

\begin{Shaded}
\begin{Highlighting}[]
\NormalTok{matrix1[}\DecValTok{2}\OperatorTok{:}\DecValTok{4}\NormalTok{, }\DecValTok{1}\OperatorTok{:}\DecValTok{3}\NormalTok{] }\CommentTok{# calls up matrix 1, columns 2 to 4 by rows 1 to 3}
\end{Highlighting}
\end{Shaded}

\begin{verbatim}
##      [,1] [,2] [,3]
## [1,]    2    7   12
## [2,]    3    8   13
## [3,]    4    9   14
\end{verbatim}

\begin{Shaded}
\begin{Highlighting}[]
\NormalTok{matrix3}
\end{Highlighting}
\end{Shaded}

\begin{verbatim}
##        un deux trois cat
## uno     1    2     3   4
## dos     5    6     7   8
## tres    9   10    11  12
## cuatro 13   14    15  16
## cinco  17   18    19  20
\end{verbatim}

\begin{Shaded}
\begin{Highlighting}[]
\NormalTok{matrix3[}\KeywordTok{c}\NormalTok{(}\DecValTok{12}\NormalTok{,}\DecValTok{14}\NormalTok{)] }\CommentTok{# get 7 and 15 because in positions 12 and 14, these are the numbers in those positions}
\end{Highlighting}
\end{Shaded}

\begin{verbatim}
## [1]  7 15
\end{verbatim}

\begin{Shaded}
\begin{Highlighting}[]
\NormalTok{cells <-}\StringTok{ }\KeywordTok{c}\NormalTok{(}\DecValTok{1}\NormalTok{, }\DecValTok{26}\NormalTok{, }\DecValTok{24}\NormalTok{, }\DecValTok{68}\NormalTok{)}
\NormalTok{rnames <-}\StringTok{ }\KeywordTok{c}\NormalTok{(}\StringTok{"R1"}\NormalTok{, }\StringTok{"R2"}\NormalTok{)}
\NormalTok{cnames <-}\StringTok{ }\KeywordTok{c}\NormalTok{(}\StringTok{"C1"}\NormalTok{, }\StringTok{"C2"}\NormalTok{) }
\NormalTok{matrix4 <-}\StringTok{ }\KeywordTok{matrix}\NormalTok{(cells, }\DataTypeTok{nrow =} \DecValTok{2}\NormalTok{, }\DataTypeTok{ncol =} \DecValTok{2}\NormalTok{, }\DataTypeTok{byrow =} \OtherTok{TRUE}\NormalTok{,}
  \DataTypeTok{dimnames =} \KeywordTok{list}\NormalTok{(rnames, cnames)) }\CommentTok{# }
\NormalTok{matrix4}
\end{Highlighting}
\end{Shaded}

\begin{verbatim}
##    C1 C2
## R1  1 26
## R2 24 68
\end{verbatim}

\begin{Shaded}
\begin{Highlighting}[]
\CommentTok{#allows you to build a matrix from scratch. line 60, call up cells, a character vector named rnames and one named cnames}
\CommentTok{# values that will go into the matrix are "cells" by 2 rows and 2 columns, and numbers arranged across rows}
\CommentTok{# add the names by including "dimnames = list(name1, name2)"}

\CommentTok{# Lists ---- }
\NormalTok{list1 <-}\StringTok{ }\KeywordTok{list}\NormalTok{(}\DataTypeTok{name =} \StringTok{"Maria"}\NormalTok{, }\DataTypeTok{mynumbers =}\NormalTok{ vector1, }\DataTypeTok{mymatrix =}\NormalTok{ matrix1, }\DataTypeTok{age =} \FloatTok{5.3}\NormalTok{); list1}
\end{Highlighting}
\end{Shaded}

\begin{verbatim}
## $name
## [1] "Maria"
## 
## $mynumbers
## [1]  1.0  2.0  5.3  6.0 -2.0  4.0
## 
## $mymatrix
##      [,1] [,2] [,3] [,4]
## [1,]    1    6   11   16
## [2,]    2    7   12   17
## [3,]    3    8   13   18
## [4,]    4    9   14   19
## [5,]    5   10   15   20
## 
## $age
## [1] 5.3
\end{verbatim}

\begin{Shaded}
\begin{Highlighting}[]
\NormalTok{list1[[}\DecValTok{2}\NormalTok{]]}
\end{Highlighting}
\end{Shaded}

\begin{verbatim}
## [1]  1.0  2.0  5.3  6.0 -2.0  4.0
\end{verbatim}

\begin{Shaded}
\begin{Highlighting}[]
\CommentTok{# list has a name called maria, a numeric vector called vector1, a matrix, and an age}
\CommentTok{# putting in a semicolon allows you to combine the two codes so you only have to run 1 line of code}
\CommentTok{# this list function lists out the separate elements of the same list}
\CommentTok{# line 74 calls up the 2nd element in list 1}
\CommentTok{# pulling from the vectors and matrices we made above. have to make those first to do the list}

\CommentTok{# Data Frames ----}
\NormalTok{d <-}\StringTok{ }\KeywordTok{c}\NormalTok{(}\DecValTok{1}\NormalTok{, }\DecValTok{2}\NormalTok{, }\DecValTok{3}\NormalTok{, }\DecValTok{4}\NormalTok{) }\CommentTok{# What type of vector? numerical}
\NormalTok{e <-}\StringTok{ }\KeywordTok{c}\NormalTok{(}\StringTok{"red"}\NormalTok{, }\StringTok{"white"}\NormalTok{, }\StringTok{"red"}\NormalTok{, }\OtherTok{NA}\NormalTok{) }\CommentTok{# What type of vector? character}
\NormalTok{f <-}\StringTok{ }\KeywordTok{c}\NormalTok{(}\OtherTok{TRUE}\NormalTok{, }\OtherTok{TRUE}\NormalTok{, }\OtherTok{TRUE}\NormalTok{, }\OtherTok{FALSE}\NormalTok{) }\CommentTok{# What type of vector? logical}
\NormalTok{dataframe1 <-}\StringTok{ }\KeywordTok{data.frame}\NormalTok{(d,e,f) }\CommentTok{# makes a data frame with the 3 objects called dataframe1}
\CommentTok{# vectors show up in values section of environment and the matrices show up in data}
\CommentTok{# R recognizes NA as its own thing, so don't need to put it in quotes}
\CommentTok{# turns character vector into factor-type of data frame. in Env says "factor w/2 levels...1 2 1 NA"}
\KeywordTok{names}\NormalTok{(dataframe1) <-}\StringTok{ }\KeywordTok{c}\NormalTok{(}\StringTok{"ID"}\NormalTok{,}\StringTok{"Color"}\NormalTok{,}\StringTok{"Passed"}\NormalTok{); }\KeywordTok{View}\NormalTok{(dataframe1) }\CommentTok{# can rename column titles using this format}
\end{Highlighting}
\end{Shaded}

\begin{verbatim}
## Warning in system2("/usr/bin/otool", c("-L", shQuote(DSO)), stdout = TRUE):
## running command ''/usr/bin/otool' -L '/Library/Frameworks/R.framework/Resources/
## modules/R_de.so'' had status 1
\end{verbatim}

\begin{Shaded}
\begin{Highlighting}[]
\NormalTok{dataframe1[}\DecValTok{1}\OperatorTok{:}\DecValTok{2}\NormalTok{,] }\CommentTok{# 2 dimensions, but each column has a different mode included in single data frame}
\end{Highlighting}
\end{Shaded}

\begin{verbatim}
##   ID Color Passed
## 1  1   red   TRUE
## 2  2 white   TRUE
\end{verbatim}

\begin{Shaded}
\begin{Highlighting}[]
\CommentTok{# matric subsetting is always row, columns}
\NormalTok{dataframe1[}\KeywordTok{c}\NormalTok{(}\StringTok{"ID"}\NormalTok{,}\StringTok{"Passed"}\NormalTok{)] }\CommentTok{# calls up specific elements of data frame, i.e. "ID" and "Passed"}
\end{Highlighting}
\end{Shaded}

\begin{verbatim}
##   ID Passed
## 1  1   TRUE
## 2  2   TRUE
## 3  3   TRUE
## 4  4  FALSE
\end{verbatim}

\begin{Shaded}
\begin{Highlighting}[]
\NormalTok{dataframe1}\OperatorTok{$}\NormalTok{ID }\CommentTok{# Just calls up individual column}
\end{Highlighting}
\end{Shaded}

\begin{verbatim}
## [1] 1 2 3 4
\end{verbatim}

Question: How do the different types of data appear in the Environment
tab?

\begin{quote}
Answer: Data frames, lists, and matrices are located in the Data tabs,
whereas, the objects like cells, names of rows and columns, and vectors
are under the Values tab.
\end{quote}

Question: In the R chunk below, write ``dataframe1\$''. Press
\texttt{tab} after you type the dollar sign. What happens?

\begin{quote}
Answer: It lists the name of the columns in dataframe1 and pushing enter
calls up the last object that was called up before
\end{quote}

\subsubsection{Coding challenge}\label{coding-challenge}

Find a ten-day forecast of temperatures (Fahrenheit) for Durham, North
Carolina. Create two vectors, one representing the high temperature on
each of the ten days and one representing the low.

\begin{Shaded}
\begin{Highlighting}[]
\NormalTok{vectorhighf <-}\StringTok{ }\KeywordTok{c}\NormalTok{ (}\DecValTok{62}\NormalTok{, }\DecValTok{45}\NormalTok{, }\DecValTok{50}\NormalTok{, }\DecValTok{50}\NormalTok{, }\DecValTok{40}\NormalTok{,}\DecValTok{40}\NormalTok{, }\DecValTok{45}\NormalTok{, }\DecValTok{52}\NormalTok{,}\DecValTok{54}\NormalTok{, }\DecValTok{58}\NormalTok{)}
\NormalTok{vectorhighf}
\end{Highlighting}
\end{Shaded}

\begin{verbatim}
##  [1] 62 45 50 50 40 40 45 52 54 58
\end{verbatim}

\begin{Shaded}
\begin{Highlighting}[]
\NormalTok{vectorlowf <-}\StringTok{ }\KeywordTok{c}\NormalTok{(}\DecValTok{31}\NormalTok{, }\DecValTok{29}\NormalTok{, }\DecValTok{43}\NormalTok{, }\DecValTok{24}\NormalTok{, }\DecValTok{23}\NormalTok{, }\DecValTok{23}\NormalTok{, }\DecValTok{26}\NormalTok{, }\DecValTok{30}\NormalTok{, }\DecValTok{42}\NormalTok{, }\DecValTok{41}\NormalTok{)}
\NormalTok{vectorlowf}
\end{Highlighting}
\end{Shaded}

\begin{verbatim}
##  [1] 31 29 43 24 23 23 26 30 42 41
\end{verbatim}

Now, create two additional vectors that include the ten-day forecast for
the high and low temperatures in Celsius.

\begin{Shaded}
\begin{Highlighting}[]
\NormalTok{vectorhighc <-}\StringTok{ }\KeywordTok{c}\NormalTok{(}\DecValTok{17}\NormalTok{, }\DecValTok{7}\NormalTok{, }\DecValTok{10}\NormalTok{, }\DecValTok{10}\NormalTok{, }\DecValTok{4}\NormalTok{, }\DecValTok{4}\NormalTok{, }\DecValTok{7}\NormalTok{, }\DecValTok{11}\NormalTok{, }\DecValTok{12}\NormalTok{, }\DecValTok{14}\NormalTok{) }\CommentTok{# numeric vector}
\NormalTok{vectorhighc}
\end{Highlighting}
\end{Shaded}

\begin{verbatim}
##  [1] 17  7 10 10  4  4  7 11 12 14
\end{verbatim}

\begin{Shaded}
\begin{Highlighting}[]
\NormalTok{vectorlowc <-}\StringTok{ }\KeywordTok{c}\NormalTok{(}\OperatorTok{-}\DecValTok{1}\NormalTok{, }\OperatorTok{-}\DecValTok{1}\NormalTok{, }\DecValTok{6}\NormalTok{, }\OperatorTok{-}\DecValTok{4}\NormalTok{, }\OperatorTok{-}\DecValTok{4}\NormalTok{, }\OperatorTok{-}\DecValTok{5}\NormalTok{, }\OperatorTok{-}\DecValTok{3}\NormalTok{, }\OperatorTok{-}\DecValTok{1}\NormalTok{, }\DecValTok{5}\NormalTok{, }\DecValTok{5}\NormalTok{) }
\NormalTok{vectorlowc}
\end{Highlighting}
\end{Shaded}

\begin{verbatim}
##  [1] -1 -1  6 -4 -4 -5 -3 -1  5  5
\end{verbatim}

\begin{Shaded}
\begin{Highlighting}[]
\NormalTok{vectorhighc}\OperatorTok{*}\NormalTok{(}\OperatorTok{-}\DecValTok{32}\OperatorTok{*}\NormalTok{(}\DecValTok{5}\OperatorTok{/}\DecValTok{9}\NormalTok{))}
\end{Highlighting}
\end{Shaded}

\begin{verbatim}
##  [1] -302.22222 -124.44444 -177.77778 -177.77778  -71.11111  -71.11111
##  [7] -124.44444 -195.55556 -213.33333 -248.88889
\end{verbatim}

Combine your four vectors into a data frame and add informative column
names.

\begin{Shaded}
\begin{Highlighting}[]
\NormalTok{dataframetemps <-}\StringTok{ }\KeywordTok{data.frame}\NormalTok{(vectorhighf, vectorlowf, vectorhighc, vectorlowc) }
\KeywordTok{names}\NormalTok{(dataframetemps) <-}\StringTok{ }\KeywordTok{c}\NormalTok{(}\StringTok{"High Temp F"}\NormalTok{,}\StringTok{"Low Temp F"}\NormalTok{,}\StringTok{"High Temp C"}\NormalTok{, }\StringTok{"Low Temp C"}\NormalTok{); }\KeywordTok{View}\NormalTok{(dataframetemps) }
\end{Highlighting}
\end{Shaded}

\begin{verbatim}
## Warning in system2("/usr/bin/otool", c("-L", shQuote(DSO)), stdout = TRUE):
## running command ''/usr/bin/otool' -L '/Library/Frameworks/R.framework/Resources/
## modules/R_de.so'' had status 1
\end{verbatim}

\begin{Shaded}
\begin{Highlighting}[]
\NormalTok{dataframetemps[}\DecValTok{1}\OperatorTok{:}\DecValTok{2}\NormalTok{,] }
\end{Highlighting}
\end{Shaded}

\begin{verbatim}
##   High Temp F Low Temp F High Temp C Low Temp C
## 1          62         31          17         -1
## 2          45         29           7         -1
\end{verbatim}

\begin{Shaded}
\begin{Highlighting}[]
\NormalTok{dataframetemps[}\KeywordTok{c}\NormalTok{(}\StringTok{"High Temp F"}\NormalTok{,}\StringTok{"High Temp C"}\NormalTok{)] }\CommentTok{# calls up specific elements of data frame, i.e. "ID" and "Passed"}
\end{Highlighting}
\end{Shaded}

\begin{verbatim}
##    High Temp F High Temp C
## 1           62          17
## 2           45           7
## 3           50          10
## 4           50          10
## 5           40           4
## 6           40           4
## 7           45           7
## 8           52          11
## 9           54          12
## 10          58          14
\end{verbatim}

\begin{Shaded}
\begin{Highlighting}[]
\NormalTok{dataframetemps}\OperatorTok{$}\StringTok{`}\DataTypeTok{High Temp F}\StringTok{`}
\end{Highlighting}
\end{Shaded}

\begin{verbatim}
##  [1] 62 45 50 50 40 40 45 52 54 58
\end{verbatim}

Use the common functions \texttt{summary} and \texttt{sd} to obtain
basic data summaries of the ten-day forecast. How would you call these
functions differently for the entire data frame vs.~a single column?
Attempt to demonstrate both options below.

\begin{Shaded}
\begin{Highlighting}[]
\KeywordTok{summary.data.frame}\NormalTok{(dataframetemps)}
\end{Highlighting}
\end{Shaded}

\begin{verbatim}
##   High Temp F     Low Temp F    High Temp C      Low Temp C   
##  Min.   :40.0   Min.   :23.0   Min.   : 4.00   Min.   :-5.00  
##  1st Qu.:45.0   1st Qu.:24.5   1st Qu.: 7.00   1st Qu.:-3.75  
##  Median :50.0   Median :29.5   Median :10.00   Median :-1.00  
##  Mean   :49.6   Mean   :31.2   Mean   : 9.60   Mean   :-0.30  
##  3rd Qu.:53.5   3rd Qu.:38.5   3rd Qu.:11.75   3rd Qu.: 3.50  
##  Max.   :62.0   Max.   :43.0   Max.   :17.00   Max.   : 6.00
\end{verbatim}

\subsection{Packages}\label{packages}

The Packages tab in the notebook stores the packages that you have saved
in your system. A checkmark next to each package indicates whether the
package has been loaded into your current R session. Given that R is an
open source software, users can create packages that have specific
functionalities, with complicated code ``packaged'' into a simple
commands.

If you want to use a specific package that is not in your libaray
already, you need to install it. You can do this in two ways:

\begin{enumerate}
\def\labelenumi{\arabic{enumi}.}
\item
  Click the install button in the packages tab. Type the package name,
  which should autocomplete below (case matters). Make sure to check
  ``intall dependencies,'' which will also install packages that your
  new package uses.
\item
  Type \texttt{install.packages("packagename")} into your R chunk or
  console. It will then appear in your packages list. You only need to
  do this once.
\end{enumerate}

If a package is already installed, you will need to load it every
session. You can do this in two ways:

\begin{enumerate}
\def\labelenumi{\arabic{enumi}.}
\item
  Click the box next to the package name in the Packages tab.
\item
  Type \texttt{library(packagename)} into your R chunk or console.
\end{enumerate}

\begin{Shaded}
\begin{Highlighting}[]
\CommentTok{# comment out install commands, use only when needed and re-comment}

\KeywordTok{library}\NormalTok{(dplyr)}
\end{Highlighting}
\end{Shaded}

\begin{verbatim}
## 
## Attaching package: 'dplyr'
\end{verbatim}

\begin{verbatim}
## The following objects are masked from 'package:stats':
## 
##     filter, lag
\end{verbatim}

\begin{verbatim}
## The following objects are masked from 'package:base':
## 
##     intersect, setdiff, setequal, union
\end{verbatim}

\begin{Shaded}
\begin{Highlighting}[]
\KeywordTok{library}\NormalTok{(ggplot2)}



\CommentTok{# Some packages are umbrellas under which other packages are loaded}

\KeywordTok{library}\NormalTok{(tidyverse)}
\end{Highlighting}
\end{Shaded}

\begin{verbatim}
## ── Attaching packages ────────────────────────────────────────────────────────────────── tidyverse 1.3.0 ──
\end{verbatim}

\begin{verbatim}
## ✓ tibble  2.1.3     ✓ purrr   0.3.3
## ✓ tidyr   1.0.0     ✓ stringr 1.4.0
## ✓ readr   1.3.1     ✓ forcats 0.4.0
\end{verbatim}

\begin{verbatim}
## ── Conflicts ───────────────────────────────────────────────────────────────────── tidyverse_conflicts() ──
## x dplyr::filter() masks stats::filter()
## x dplyr::lag()    masks stats::lag()
\end{verbatim}

Question: What happens in the console when you load a package?

\begin{quote}
Answer:
\end{quote}

\subsection{Tips and Tricks}\label{tips-and-tricks}

\begin{itemize}
\item
  Sequential section headers can be created by using at least four -, =,
  and \# characters.
\item
  The command \texttt{require(packagename)} will also load a package,
  but it will not give any error or warning messages if there is an
  issue.
\item
  You may be asked to restart R when installing or updating packages.
  Feel free to say no, as this will obviously slow your progress.
  However, if the functionality of your new package isn't working
  properly, try restarting R as a first step.
\item
  If asked ``Do you want to install from sources the packages which
  needs compilation?'', type \texttt{yes} into the console.
\item
  You should only install packages once on your machine. If you store
  \texttt{install.packages} in your R chunks/scripts, comment these
  lines out.
\item
  Update your packages regularly!
\end{itemize}

\end{document}
